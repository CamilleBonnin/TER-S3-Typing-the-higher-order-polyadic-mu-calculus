\documentclass[11pt,a4paper]{article}
\usepackage[utf8]{inputenc}
\usepackage[french]{babel}
\usepackage[T1]{fontenc}
\usepackage{amsmath}
\usepackage{amsfonts}
\usepackage{amssymb}
\usepackage{makeidx}
\usepackage{graphicx}
\usepackage{lmodern}
\usepackage{ntheorem}
\usepackage{xcolor}
\usepackage{mathpartir}
\usepackage[left=2cm,right=2cm,top=2cm,bottom=2cm]{geometry}
\newcommand{\avariance}{\upsilon}
\title{Rapport (structure)}
\date{}

\begin{document}

\maketitle

\section*{Introduction}

Présentation rapide du $\mu$-calcul et de ses applications (recherches à faire regarder le model-checking). Donner l'intérêt du $\mu$-calcul par rapport à d'autres logiques dans ces applications.   

\section{Le $\mu$-calcul}

\subsection{Généralités / propriétés (titre à revoir)}

Logique modale et polyadique (rappeler définitions). 

\subsection{Syntaxe et sémantique}

\subsubsection{Enoncés}

Enoncer la syntaxe et la sémantique.

\subsubsection{Exemples}

Exemples pour mieux comprendre le sens des opérateurs. Regarder dans le(s) papier(s) ou à inventer.

\subsection{Variances}

\subsubsection{Définition et intérêt (titre à revoir)}

Intérêt pour la validité de formules avec le point fixe. 
\\
Donner une intuition de ce qu'est la variance. Essayer de donner une définition

\subsubsection{Définitions et propriétés}

Treillis des variances (expliquer le sens des variances), additivité, "N-additivité", sémantiques, règles de calcul. 

\subsubsection{Exemples}

Exemples de calculs et du formules valides et non valides à cause de la variance (cf papier).

\subsection{Règles de typage}

\subsubsection{Enoncé}

Enoncer les règles du papier + commentaires (explications).


\subsubsection{Exemples}

Donner des exemples (reprendre les précédents + prendre des exemples des formules pour les résultats), faire les arbres + commenter.

\section{($1^{ere}$ ?) Implémentation des règles de typage}

On a enlevé la règle \{i $\longleftarrow$ j\}. (explications).

\subsection{Réécriture des règles de typage}

\subsubsection{Gamma}

Expliquer le Gamma.

\subsubsection{Enoncé et justification}

Donner les nouvelles règles, faire le lien avec les anciennes.

\subsubsection{Exemples}

Reprendre les exemples de la partie précédente.

\subsection{Travail récupéré (titre à revoir)}

Diviser en plusieurs parties.
\\
Ce qui a été fait par Thomas (syntaxe, "desugar", calculs sur les variances). 
\\
Expliquer les choix pour ce qui n'a pas été gardé ou changé. 

\subsection{Fonction de typage}

\subsubsection{Implémentation des environnements de typage}

Représentation des environnements de typage complets et incomplets + des assignements. 

\subsubsection{La fonction de typage}

Pseudo-code de la fonction "typing". 
1 section par cas ? (à voir en fonction de la taille).
\\
Expliquer le code en le liant aux règles de typage réécrites. 

\subsection{Fonctions annexes}

Calcul de "inter", Gamma "rond" variance, etc : pseudo-code + explications.

\section{(Piste(s) d')Améliorations}

\subsection{Les limites de l'implémentation actuelle}

Le Gamma. 

\subsection{(Piste(s) d')Améliorations}

A voir.

\section{Résultats}

\subsection{Tableau des résultats}

Faire un tableau avec les résultats de tests.

\subsection{Commentaires}

Commenter les résultats, peut-être faire des arbres de typage pour certains.

\section*{Conclusion}

Bref résumé + avis personnel + piste(s) d'approfondissement (adapter avec la partie 3).


\bibliographystyle{splncs04}
\bibliography{TerS3}
\end{document}
