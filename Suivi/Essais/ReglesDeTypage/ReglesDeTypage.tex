\documentclass[11pt,a4paper]{article}
\usepackage[utf8]{inputenc}
\usepackage[french]{babel}
\usepackage[T1]{fontenc}
\usepackage{amsmath}
\usepackage{amsfonts}
\usepackage{amssymb}
\usepackage{makeidx}
\usepackage{graphicx}
\usepackage{lmodern}
\usepackage{ntheorem}
\usepackage{xcolor}
\usepackage{mathpartir}
\usepackage[left=2cm,right=2cm,top=2cm,bottom=2cm]{geometry}
\newcommand{\avariance}{\upsilon}
\title{Règles de typage du $\mu$-calcul }

\begin{document}

\begin{center}
  \Huge{\textbf{Règles de typage du $\mu$-calcul}}\\[1cm]
\end{center}

\!\!\!\!\!\!\!\!\!\!\Large {\textbf{Règles actuelles  \cite{LANGE2014326}}}\\

\begin{center}
$\inferrule
{ }
{\Gamma \vdash \top $ : $\bullet}
\qquad
\inferrule
{\Gamma \vdash \Phi_{1} : \bullet \\ \Gamma \vdash \Phi_{2} : \bullet}
{\Gamma \vdash \Phi_{1} \wedge \Phi_{2} : \bullet}
\qquad
\inferrule
{\overline{\Gamma} \vdash \Phi : \tau}
{\Gamma \vdash \neg\Phi : \tau}$
\\
$ $
\\
$ $
\\
$\inferrule
{\Gamma \vdash \Phi : \bullet}
{\{\sqcup\} \circ \Gamma \vdash \langle a \rangle_{i} \Phi : \bullet}
\qquad
\inferrule
{\Gamma \vdash \Phi : \bullet}
{\Gamma \vdash \{\vec{i} \leftarrow \vec{j}\} \Phi : \bullet}
\qquad
\inferrule
{\upsilon \subseteq \{\sqcap, \sqcup\}$ or $\upsilon = any}
{\Gamma, X^{\upsilon} : \tau \vdash X : \tau}$
\\
$ $
\\
$ $
\\
$
\inferrule
{\Gamma_{1} \vdash \mathfrak{F} : \bullet^{\upsilon} \rightarrow \tau \\
\Gamma_{2} \vdash \Phi : \bullet}
{\Gamma \vdash \mathfrak{F} \Phi : \tau}
\enspace {}^{\Gamma \: \preceq \: \Gamma_{1}}_{\Gamma \: \preceq \: \upsilon \: \circ \: \Gamma_{2}}
$
\\
$ $
\\
$ $
\\
$\inferrule
{\Gamma, X^{\avariance} : \tau \vdash \Phi}
{\Gamma \vdash \mu X : \tau \: . \: \Phi}
\enspace X \notin vars(\Gamma)
\qquad
\inferrule
{\Gamma, X^{\upsilon} : \bullet \vdash \Phi : \tau}
{\Gamma \vdash \lambda X^{\upsilon} : \bullet \: . \: \Phi : \bullet^{\upsilon} \rightarrow \tau}
\enspace X \notin vars(\Gamma)$
\end{center}
$ $
\\
\Large {\textbf{Nouvelles règles}}\\

On pose $type(\Delta,f) = (\Gamma, \tau)$ avec f, une formule écrite selon les règles du $\mu$-calcul, $\Gamma$, l'environnement de typage de f et $\tau$, le type de f. $\Delta$ représente un environnement de typage incomplet, sans les variances des variables, par exemple $\Delta=(X:\bullet,Y:\bullet^{\emptyset}\to\bullet)$. Notons que par contre les types qui apparaissent dans $\Delta$ ont les informations de variabnce. Par la suite, on essaiera de se passer de ces informations de variance, voire de $\Delta$ en entier... Pour le moment on suppose ce $\Delta$ donné, et ceci grace à des annotations de typage dans la formule que l'on cherche à typer.
Seules les annotations de variance pour les variables introduites par $\lambda$ ne sont pas données.
Par exemple une formule $f$ possible est $\lambda X:\bullet^{\emptyset}\to\bullet.\ X\top$.
\\
TODO: rentre tout cela plus formel en donnant une grammaire pour $f$, pour $\Delta$, pour $\tau$,
et pour $\Gamma$.


Dans la suite, $\emptyset$ représente l'environnement de typage vide.

On note $\avariance_1\wedge\avariance_2$ la variance ``inf'' des deux variances $\avariance_1$ et $\avariance_2$ au sens du
treillis des variances. On note $\Gamma_1\wedge\Gamma_2$ l'environnement de typage tel que :
\begin{itemize}
\item si $X^{\avariance}:\tau$
apparait dans $\Gamma_1$ et $X\not\in\mathsf{vars}(\Gamma_2)$, alors $X^{\avariance}:\tau$ apparait dans $\Gamma_1\wedge\Gamma_2$
\item si $X^{\avariance}:\tau$
apparait dans $\Gamma_2$ et $X\not\in\mathsf{vars}(\Gamma_1)$, alors $X^{\avariance}:\tau$ apparait dans $\Gamma_1\wedge\Gamma_2$
\item si $X^{\avariance_1}:\tau_1$ apparait dans $\Gamma_1$ et $X^{\avariance_2}:\tau_2$ apparait dans $\Gamma_2$,
alors (1) si $\tau_1\neq\tau_2$, $\Gamma_1\wedge\Gamma_2$ n'est pas défini, et (2) si $\tau_1=\tau_2$, alors
$X^{\avariance_1\wedge\avariance_2}:\tau_1$ apparait dans $\Gamma_1\wedge\Gamma_2$.
\end{itemize}
\begin{center}
$\inferrule
{ }
{type(\Delta, \top) = (\emptyset, \bullet)}
\qquad
\inferrule
{type(\Delta,\Phi_{1}) = (\Gamma_{1}, \bullet) \\ type(\Delta,\Phi_{2}) = (\Gamma_{2}, \bullet)}
{type(\Delta,\Phi_{1} \wedge \Phi_{2}) = (\Gamma_{1} \wedge \Gamma_{2}, \bullet)}$
\\
$ $
\\
$ $
\\
$\inferrule
{type(\Delta,\Phi) = (\Gamma, \tau)}
{type(\Delta,\neg\Phi) = (\overline{\{\sqcap, \sqcup\}} \circ \Gamma, \tau)}
\qquad
\inferrule
{type(\Delta,\Phi) = (\Gamma, \bullet)}
{type(\Delta,\langle a \rangle_{i} \Phi) = (\{\sqcup\} \circ \Gamma, \bullet)}$
\\
$ $
\\
$ $
\\
$\inferrule
{type(\Delta,\Phi) = (\Gamma, \bullet)}
{type(\Delta,\{\vec{i} \leftarrow \vec{j}\} \Phi) = (\Gamma, \bullet)}
\qquad
\inferrule
{ (X:\tau)\in\Delta}
{type(\Delta,X) = (X^{\{\sqcap,\sqcup\}}:\tau, \tau)}$
\\
$ $
\\
$ $
\\
$\inferrule
{type(\Delta,\mathfrak{F}) = (\Gamma_{1}, \sigma^{\avariance} \rightarrow \tau) \\ type(\Delta,\Phi) = (\Gamma_{2}, \sigma)}
{type(\Delta,\mathfrak{F} \Phi) = (\Gamma_1\wedge\avariance\circ\Gamma_2, \tau)}
$
\\
$\inferrule
{type(\Delta\cup\{X:\tau\},\Phi)=(\Gamma,\sigma)\\ \sigma=\tau
\\ \Gamma=\Gamma'\cup\{X^{\avariance}:\sigma\} \\ \avariance \succeq \varnothing}
{type(\Delta,\mu X : \tau \: . \: \Phi)=(\Gamma',\tau)}
$
\\
$
\inferrule
{type(\Delta\cup\{X:\sigma\},\Phi)=(\Gamma,\tau) \\ \Gamma=\Gamma'\cup\{X^{\avariance}: \sigma'\}}
{type(\Delta,\lambda X : \sigma \: . \: \Phi)= (\Gamma',\sigma'^{\avariance} \rightarrow \tau)}
$
\\
$ $
\end{center}

\bibliographystyle{splncs04}
\bibliography{TerS3}
\end{document}
