\documentclass[11pt,a4paper]{article}
\usepackage[utf8]{inputenc}
\usepackage[french]{babel}
\usepackage[T1]{fontenc}
\usepackage{amsmath}
\usepackage{amsfonts}
\usepackage{amssymb}
\usepackage{makeidx}
\usepackage{graphicx}
\usepackage{lmodern}
\usepackage{ntheorem}
\usepackage{xcolor}
\usepackage{mathpartir}
\usepackage[left=2cm,right=2cm,top=2cm,bottom=2cm]{geometry}
\title{Règles de typage du $\mu$-calcul }

\begin{document}

\begin{center}
  \Huge{\textbf{Règles de typage du $\mu$-calcul}}\\[1cm]
\end{center}
  
\!\!\!\!\!\!\!\!\!\!\Large {\textbf{Règles actuelles  \cite{LANGE2014326}}}\\

\begin{center}
$\inferrule 
{ } 
{\Gamma \vdash \top $ : $\bullet} 
\qquad 
\inferrule 
{\Gamma \vdash \Phi_{1} : \bullet \\ \Gamma \vdash \Phi_{2} : \bullet} 
{\Gamma \vdash \Phi_{1} \wedge \Phi_{2} : \bullet}
\qquad
\inferrule 
{\overline{\Gamma} \vdash \Phi : \tau} 
{\Gamma \vdash \neg\Phi : \tau}$
\\
$ $
\\
$ $
\\
$\inferrule 
{\Gamma \vdash \Phi : \bullet} 
{\{\sqcup\} \circ \Gamma \vdash \langle a \rangle_{i} \Phi : \bullet}
\qquad
\inferrule 
{\Gamma \vdash \Phi : \bullet} 
{\Gamma \vdash \{\vec{i} \leftarrow \vec{j}\} \Phi : \bullet}
\qquad
\inferrule 
{\upsilon \subseteq \{\sqcap, \sqcup\}$ or $\upsilon = any} 
{\Gamma, X^{\upsilon} : \tau \vdash X : \tau}$
\\
$ $
\\
$ $
\\
$
\inferrule 
{\Gamma_{1} \vdash \mathfrak{F} : \bullet^{\upsilon} \rightarrow \tau \\
\Gamma_{2} \vdash \Phi : \bullet} 
{\Gamma \vdash \mathfrak{F} \Phi : \tau}
\enspace {}^{\Gamma \: \preceq \: \Gamma_{1}}_{\Gamma \: \preceq \: \upsilon \: \circ \: \Gamma_{2}}
$
\\
$ $
\\
$ $
\\
$\inferrule 
{\Gamma, X^{\varnothing} : \tau \vdash \Phi} 
{\Gamma \vdash \mu X : \tau \: . \: \Phi}
\enspace X \notin vars(\Gamma)
\qquad
\inferrule 
{\Gamma, X^{\upsilon} : \bullet \vdash \Phi : \tau} 
{\Gamma \vdash \lambda X^{\upsilon} : \bullet \: . \: \Phi : \bullet^{\upsilon} \rightarrow \tau}
\enspace X \notin vars(\Gamma)$
\end{center}
$ $
\\
\Large {\textbf{Nouvelles règles}}\\

On pose type(f) = ($\Gamma$, $\tau$) avec f, une formule écrite selon les règles du $\mu$-calcul, $\Gamma$, l'environnement de typage de f et $\tau$, le type de f.
\\
Dans la suite, $\emptyset$ représente l'environnement de typage vide.
  
\begin{center}
$\inferrule 
{ } 
{type(\top) = (\emptyset, \bullet)} 
\qquad 
\inferrule 
{type(\Phi_{1}) = (\Gamma_{1}, \bullet) \\ type(\Phi_{2}) = (\Gamma_{2}, \bullet)} 
{type(\Phi_{1} \wedge \Phi_{2}) = (\Gamma_{1} \cap \Gamma_{2}, \bullet)}\: ???$
\\
$ $
\\
$ $
\\
$\inferrule 
{type(\Phi) = (\Gamma, \tau)} 
{type(\neg\Phi) = (\overline{\{\sqcap, \sqcup\}} \circ \Gamma, \tau)}
\qquad
\inferrule 
{type(\Phi) = (\Gamma, \bullet)} 
{type(\langle a \rangle_{i} \Phi) = (\{\sqcup\} \circ \Gamma, \bullet)}$
\\
$ $
\\
$ $
\\
$\inferrule 
{type(\Phi) = (\Gamma, \bullet)} 
{type(\{\vec{i} \leftarrow \vec{j}\} \Phi) = (\Gamma, \bullet)}
\qquad
\inferrule 
{\upsilon \subseteq \{\sqcap, \sqcup\}$ or $\upsilon = any}
{type(X) = (\upsilon, \tau)}\: ???$
\\
$ $
\\
$ $
\\
$\inferrule 
{\inferrule
{type(e) = (\upsilon, \bullet)}
{type(\mathfrak{F}) = (\Gamma_{1}, e \rightarrow \tau)} \\ type(\Phi) = (\Gamma_{2}, \bullet)}
{type(\mathfrak{F} \Phi) = (\Gamma, \tau)}
\: \enspace {}^{\Gamma \: \preceq \: \Gamma_{1}}_{\Gamma \: \preceq \: \upsilon \: \circ \: \Gamma_{2}} ???$
\\
$ $
\\
$ $
\\
$\inferrule 
{\Gamma, X^{\varnothing} : \tau \vdash \Phi} 
{\Gamma \vdash \mu X : \tau \: . \: \Phi}
\enspace X \notin vars(\Gamma)
\qquad
\inferrule 
{\Gamma, X^{\upsilon} : \bullet \vdash \Phi : \tau} 
{\Gamma \vdash \lambda X^{\upsilon} : \bullet \: . \: \Phi : \bullet^{\upsilon} \rightarrow \tau}
\enspace X \notin vars(\Gamma)$
\\
$ $
\end{center}
	
\bibliographystyle{splncs04}
\bibliography{TerS3}
\end{document}